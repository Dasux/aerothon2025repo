\documentclass[12pt]{report}
\usepackage[a4paper, margin=0.5in]{geometry}
\usepackage{tikz, xcolor, graphicx, wrapfig, hyperref, amsmath, amssymb, geometry, array, float}
\renewcommand{\familydefault}{\sfdefault}
\usepackage[scaled]{helvet}
\usepackage[table]{xcolor}
\geometry{margin=1in}


\usepackage{titlesec}

\titleformat{\chapter}[display]
  {\normalfont\huge\bfseries}{}{0pt}{\Huge\vspace{-2cm}} % you can tweak this

\titlespacing*{\chapter}{0pt}{5pt}{20pt} % left, before, after


\begin{document}

% ====================== TITLE ========================
  \begin{center}
    % \vspace*{-1in}
    \includegraphics[width=0.4\textwidth]{sae_aerospace.png}\\
    \vspace{1cm}
    \textbf{\LARGE AEROTHON 2025} \\
    \vspace{0.5in}
    \textbf{\Large UNCREWED AIRCRAFT SYSTEM  (UAS) \\
    DESIGN, BUILD \\ AND FLY CONTEST}\\
    \vspace{0.5cm}
    \includegraphics[width=0.4\textwidth]{aerothon2025.png}\\
    \vspace{0.5cm}
    \Large \textbf{PHASE 1:} DESIGN REPORT \\
    \vspace{0.5cm}
    \Large \textbf{TEAM NAME:} UDSAV \\
    \Large \textbf{TEAM NUMBER: } AT2025043
    
    \vspace{0.5cm}
    \includegraphics[width=0.3\textwidth]{gsvlogo.png}\\
    \vspace{0.5cm}
    \textbf{\Large GATI SHAKTI VISHWAVIDYALAYA} \\
    \large{Ministry of Railways, Govt. of India \\Vadodara, India - 390004}
  \end{center}
  % ======================================================
  \newpage
  \begin{figure}[h] 
    \centering
    \includegraphics[width=1\textwidth]{certificate_faculty.jpeg}
  \end{figure}
  \newpage

  \newgeometry{left=1in, right=0.5in, top=1in, bottom=0.5in}
  % ============ [TABLE OF CONTENTS] ================
  \tableofcontents
  \newpage

  \newgeometry{left=1in, right=0.5in, top=0.5in, bottom=0.5in}
  \renewcommand{\arraystretch}{1.5}
      \rowcolors{1}{white}{blue!10}

  % ================== Introduction =================
  \chapter{Introduction}
  \section{Overview}

   In the face of natural and man-made disasters, rapid response and situational awareness are critical. Drones have emerged as powerful tools in disaster management, offering real-time aerial insights, access to hard-to-reach areas, and faster deployment compared to traditional methods. Whether locating survivors, assessing damage, or delivering essential supplies, drone technology enhances the efficiency and safety of relief operations. As disasters grow more complex and unpredictable, integrating drones into emergency response systems is no longer a luxury—it's a necessity. \\ 

   Through Aerothon, Team UDSAV (\textit{Uncrewed Disaster Surveillance Aerial Vehicle}) is not just competing—we are contributing to the evolution of drone-assisted disaster response, pushing the boundaries of what UAVs can achieve in life-saving missions.

    \section{Problem Statement and Mission Requirements}
    This year's AEROTHON is themed on \textbf{\textit{Surveillance and Disaster Management.}} The problem statement is to build an \textbf{\textit{Uncrewed Aircraft System (UAS)}} to be able to perform the mission requirements as per the rulebook. The mission requirements at a glance are as follows: \\ \\
    \textbf{Mission - 1:} \textit{Advanced Obstacle Navigation $\&$ Fragile Payload Delivery with Precision Placement – Manual Operation} \\ \\
    \textbf{Mission - 2:} \textit{Autonomous Object Classification, Disaster Situation Identification $\&$ Payload Drop – Autonomous Operation} 



    \section{Scope of Report}
    The scope of this report is to provide a comprehensive understanding of the design rationale we have used while building this project. We have tried to provide the relevant calculations, figures, and analysis models to justify the materials/design/framework we've chosen to work with for our structural and system architectures. \\ 

    \noindent Apart from that, this report is intended to also serve as an accessible guide catering to neophytes in UAV/UAS systems. We have tried our best to aim at providing clear context and insight that sort of demystifies drone development. \\

  % ======== Requirements and Design Objectives =========
  \section{System Requirements \& Design Objectives}
    \subsection{Mission Profile}
    \begin{enumerate}
      \item \textbf{Mission 1:} \textit{Advanced Obstacle Navigation $\&$ Fragile Payload Delivery with Precision Placement} \\
        This is a \textbf{\textit{Manual Operation.}} In this mission, the drone must transport a fragile payload through a challenging course filled with static obstacles such as walls, barriers, and narrow passages. The primary objective is to navigate these obstacles with high precision while ensuring the payload remains undamaged. \\
        
        Upon reaching the target zone, the drone must land carefully and place the fragile payload on the ground without causing any damage. After the successful placement, the drone must then return to the takeoff point or designated home base, ensuring safe and efficient navigation back through the course. The mission is complete once the payload is placed securely, and the drone successfully returns to the home base.
      \item \textbf{Mission 2:} \textit{Autonomous Object Classification, Disaster Situation Identification $\&$ Payload Drop }\\
        This is an \textbf{\textit{Autonomous Operation.}} In this mission, the drone will autonomously scan, classify, and assess objects within a predefined area using onboard sensors and algorithms. The objects will vary in shape, size, color, and structure, and may be partially obscured, presenting challenges for detection and classification. Once the objects are classified, the drone will identify potential disaster scenarios, such as flooding, fire, or damaged infrastructure, within the same area.
    \end{enumerate}
    \subsection{Key Performance Indicators \& Constraints}
    According to the above defined mission profiles, we have a few KPIs (\textit{Key Performance Index}) to keep in mind.
    \begin{enumerate}
      \item Flight Endurance and Range 
      \item Payload Handling 
      \item Autonomous Capabilites
      \item System Reliability
      \item Design and Innovation
    \end{enumerate}
        The design and development of the UAV is subjected to several constraints as per the guidelines mentioned in the rulebook AEROTHON 2025. These include dimensional constraints, payload restrictions and strict autonomy requirements. The drone must perform all missions bound by these constraints and we have taken great time and care to articulate them down to ensure nothing is amiss.
    \begin{enumerate}
      \item \textbf{Dimensional Constraints}
        \begin{itemize}
          \item Maximum Wingspan: \textbf{1.5 metres} - the UAV must fit inside a \textbf{\textit{1.5m x 1.5m x 1.5m bounding box}} in assembled condition.
          \item Maximum Takeoff Weight: $ \boldsymbol{< 2kg} $ including battery and payload.
        \end{itemize}
      \item \textbf{Payload Constraints}
        \begin{itemize}
          \item Payload: One fragile payload cube of \textbf{\textit{12cm x 7cm x 7cm}} weighing \textbf{\textit{200g}}.
          \item Payload must be released within a \textbf{\textit{3m x 3m}} target zone.
        \end{itemize}
      \item \textbf{Flight Environment Constraints}
        \begin{itemize}
          \item Missions are conducted in \textbf{\textit{open outdoor airspace}}.
          \item Expect wind speeds upto \textbf{\textit{5m/s}}
        \end{itemize}
      \item \textbf{Autonomy and Mission Constraints}
        \begin{itemize}
          \item \textbf{Mission 1:} Manual flight only (no GPS or autopilot usage).
          \item \textbf{Mission 2:} Fully autonomous flight (no pilot intervention or RC use).
          \item All autonomous missions must avoid obstacles and make decisions based on \textbf{\textit{onboard computation.}}
        \end{itemize}
      \item \textbf{Power and Communication Constraints}
        \begin{itemize}
          \item Must operate on battery only
          \item No cellular or internet-based comms allowed
          \item Only 2.4 GHz or 5.8 GHz RF modules permitted
        \end{itemize}
      \item \textbf{Safety and Compliance}
        \begin{itemize}
          \item Must have a failsafe mode (e.g., return-to-home or emergency land)
          \item Must pass technical inspection before flying
          \item Compliance with DGCA drone guidelines (if relevant in test zones)
        \end{itemize}
      \item \textbf{Operational Constraints}
        \begin{itemize}
          \item The team must complete the flight within a \textbf{\textit{15-minute slot.}}
          \item Payload must be dropped in an area of \textbf{\textit{3m x 3m.}}
        \end{itemize}
    \end{enumerate}
  % ========= Conceptual Design Approach =======
  \chapter{Conceptual Design Approach}
    \section{Design Methodology}
    \section{Product Benchmark \& Trade-off Analysis} % why we chose this design over other 
                                                 
  % ======== Aerodynamic and Flight Performance ======
  \chapter{Detailed Design Breakdown}                                 
    \section{Preliminary Weight Estimation}
      \begin{table}[H]
      \centering
      \caption{Detailed Weight Breakdown}
        \begin{tabular}{|>{\raggedright\arraybackslash}p{6cm}|>{\raggedright\arraybackslash}p{6cm}|}
          \hline
          \textbf{Parameter} & \textbf{Weight \textit{(gms)}} \\
          \hline
          NVIDIA Jetson Orin Nano & 176 \\
          Pixhawk 2.4.8	& 39 \\
          Camera (\textit{x2})	 & 20 \\
          SpeedyBee BL32 50A 4-in-1 ESC	& 90 \\
          DYS D2836-7 1120KV BLDC (\textit{x4}) &	280 \\
          Battery	({\small Orange 5200mAh 11.1V
3S)} & 360  \\
          GPS – Neo M8N & 	23  \\
          \textcolor{red}{Transmitter (SkyDroid)} & 	\textcolor{red}{525} \\
          Receiver & 17 \\
          Payload	& 200 \\
          Additional Wiring & 50 \\
          Servo Motor & 10 \\
          Propellor (9") & 40 \\
          Estimated Frame Weight & 500 \\
          \hline
          \textbf{\textit{Total:}} & \textbf{2330} \\
          \hline
        \end{tabular}
      \end{table}
      \textcolor{red}{ \textbf{Note:} The transmitter is not a part of the UAS itself, so effective drone weight is \textbf{1805gms}. }
      
  % ============== Propulsion ====================
    \section{Thrust Requirement \& Propulsion System Selection}
      \subsection{Thrust Requirement}
      \begin{wrapfigure}{l}{0.3\textwidth}
        \includegraphics[width=0.175\textwidth]{thrust3.png}
        \caption{\small Thrust-to-Weight Diagram}
        \label{fig:thrustuas}
      \end{wrapfigure}
      
      To ensure stable and controlled flight, a multirotor drone must generate sufficient thrust to overcome its total weight. The drone in this design has a total takeoff weight of \textbf{1805 grams} (1.805 kg). For stable hovering, the combined thrust of all motors should ideally be at least equal to the total weight. However, to allow for effective maneuverability, rapid ascent, and compensation for wind or payload imbalance, a common design guideline is to target a thrust-to-weight ratio of at least \textbf{2:1}. This implies a minimum total thrust of approximately $2 \times 1.805 = 3.61$ kg. The selected propulsion system comprises four DYS D2836-7 1120KV brushless DC motors. According to manufacturer test data, when paired with a suitable 10$\times$4.7 propeller and a 3S (11.1V) LiPo battery, each motor can produce up to approximately \textbf{1130 grams} of thrust. Therefore, the total available thrust from all four motors is approximately \textbf{4.52 kg}, yielding a thrust-to-weight ratio of $4.52 / 1.805 \approx 2.5$. This satisfies the performance margin and confirms that the chosen motor-propeller combination is adequate for the drone's operational requirements.

      \subsection{Motor, ESC \& Propellor}
      \subsubsection{Motor: DYS D2836-7 1120KV Brushless Motor} 
      \begin{wrapfigure}{r}{0.3\textwidth}
        \includegraphics[width=0.3\textwidth]{bldc.png}
        \caption{DYS D2836-7 1120kV BLDC}
        \label{fig:bldc1120}
      \end{wrapfigure}
      The DYS D2836-7 1120KV Brushless Motor is our go-to motor for this project because of the following leverages it offers:
        
      \begin{itemize}
        \item \textbf{\textit{KV Rating: 1120KV}} - KV generally means RPM per volt. In layman terms, in one volt, how many rotations does it make per minute = KV. In this case, 1120KV is mid-range, which means good thrust at moderate RPMs, and it works decently with larger propellors (\textit{9" - 11"}) which improves lift and efficiency, \textit{especially} at low speeds. This is perfect for surveillance drones that require loitering and stability. A lower KV would force us to use bulky propellors, and a higher KV would drain the battery faster. 1120KV is a sweet spot between the two.
        \item \textbf{\textit{Power \& Efficiency:}} - With a 3S or 4S LiPo, this motor produces ~800g to 1100g of thrust, depending on the propeller used. It can pull 20–25A max, so it's efficient for mid-weight UAVs (\textit{in our case, it is around \textbf{1.5 ~ 2kg AUW} (All Up Weight.)}), so it's ideal for our choice.
      \end{itemize}
      
      
      \begin{table}[H]
      \centering
      \caption{Motor Datasheet (DYS D2836-7 1120kV BLDC)}
        \begin{tabular}{|>{\raggedright\arraybackslash}p{6cm}|>{\raggedright\arraybackslash}p{6cm}|}
          \hline
          \textbf{Parameter} & \textbf{Value} \\
          \hline
          Motor KV (RPM/V) & 1120 \\
          \hline
          Motor Type & Brushless Motor \\
          \hline
          Compatible LiPO Batteries & 2S to 4S \\
          \hline
          Weight (g) & 70 \\
          \hline
          Shaft Diameter (mm) & $\Phi$4.0×49mm \\
          \hline
          Max. Power (W) & 336 \\
          \hline
          Maximum Thrust (gm) & 1130 \\
          \hline
          Compatible Propeller (inch) & 9×5 \\
          \hline
          Required ESC (A) & 40 \\
          \hline
        \end{tabular}
      \end{table}
      \newpage
      \subsubsection{ESC: SpeedyBee BL32 50A 4-in-1 ESC} 
      \begin{wrapfigure}{r}{0.3\textwidth}
        \includegraphics[width=1\linewidth]{esc.png}
        \caption{SpeedyBee BL32 50A 4-in-1 ESC}
        \label{fig:esc50a}
      \end{wrapfigure}

      The SpeedyBee BL32 50A 4-in-1 ESC is a good choice for surveillance drones, and our use case for the following reasons:
      \begin{itemize}
        \item \textbf{\textit{High Current Rating (50A per motor):}} Supports high-thrust motors and larger propellers. Useful for longer flight times, heavy payloads (cameras, sensors, gimbals), and stable cruising. Provides headroom — motors drawing 20–30A will run cooler and more reliably under a 50A ESC.
        \item \textbf{\textit{BLHeli\_32 Firmware:}} Smoother motor response, more efficient power delivery, and better low-end throttle control, which helps in steady hovering and slow maneuvering — perfect for surveillance.
        \item \textbf{\textit{4-in-1 Design:}} Combines 4 ESCs into one board, and reduces weight and wiring complexity. Makes the stack cleaner, ideal for modular or compact drone frames. Fewer potential failure points (vs. 4 individual ESCs).
        \item \textbf{\textit{Telemetry \& Monitoring:}} Supports ESC telemetry (RPM, current, temperature) via BLHeli\_32. This is important for diagnostics, health monitoring, and autonomous missions — ensuring no motor overheats or fails mid-flight.
        \item \textbf{\textit{Built for 3–6S LiPo:}} Offers flexibility across drone designs. For surveillance, a 4S or 6S setup is common due to higher efficiency and flight duration. This ESC handles both without issue.
        \item \textbf{\textit{Built-in TVS Protection:}} Has \textbf{Transient Voltage Suppression diodes} that protect against voltage spikes — vital for drone safety, especially in critical missions.
      \end{itemize}

      \begin{table}[H]
      \centering
      \caption{ESC Datasheet (SpeedyBee BL32 50A 4-in-1 ESC)}
        \begin{tabular}{|>{\raggedright\arraybackslash}p{6cm}|>{\raggedright\arraybackslash}p{6cm}|}
          \hline
          \textbf{Parameter} & \textbf{Value} \\
          \hline
          Continuous Current (A)	& 50\\
          PWM Frequency Range (KHz) & 16 to 128\\
          Input Power (W)	& 3-6S LiPo \\
          Current Sensor Input	& Support (Scale=490 Offset=0) \\
          Mounting Hole Diameter (mm)	& 30.5 x 30.5mm( {\small 4mm hole diameter}) \\
          Length (mm)	&          45.6\\
          Width (mm)	&      40 \\
          Height (mm)	&        8.8 \\
          Weight (g)	& 19.2g with heat sink\\
          \hline
        \end{tabular}
      \end{table}
      \vspace{0.5cm}
      \subsubsection{Propellor: HQProp Thin Electric Prop 9×5 (2CCW) Propeller}

      \begin{wrapfigure}{r}{0.3\textwidth}
        \includegraphics[width=1\linewidth]{prop.png}
        \caption{HQProp Thin Electric Prop 9×5 (2CCW) Propeller}
        \label{fig:prop9x5}
      \end{wrapfigure}

      This propeller is perfect for our use case for the following reasons:

      \begin{itemize}
        \item \textbf{\textit{High Efficiency for Long Endurance Flights:}} It is Thin electric profile = low drag → reduces current draw. Designed for cruise efficiency over brute force thrust, it serves perfect for surveillance missions where hovering and slow, steady forward flight dominate.
        \item \textbf{\textit{Optimized for Mid-Sized Motors (like D2836-7):}} The 9-inch diameter is a good disc area for smooth lift, and 5-inch pitch gives moderate speed per RPM (good forward motion without excess current). These features allows it to pair well with 1000–1200KV motors on 3S LiPo → ideal thrust-to-efficiency balance.
        \item \textbf{\textit{Smooth Throttle Response:}} Thin blades create less turbulence and vibration. This is crucial for gimbal-mounted cameras or FPV systems, reducing jello and image blur.
        \item \textbf{\textit{Expected Performance on 3S + DYS D2836-7:}} 
          \begin{enumerate}
            \item Static Thrust \hfill 850--1000g
            \item Current @ full throttle \hfill 15--18A
            \item Thrust Efficiency \hfill $\approx$ 60--65 g/W
          \end{enumerate}
      \end{itemize}
      \begin{table}[H]
      \centering
      \caption{Propeller Datasheet (HQProp Thin Electric Prop 9×5 (2CCW) Propeller)}
        \begin{tabular}{|>{\raggedright\arraybackslash}p{6cm}|>{\raggedright\arraybackslash}p{6cm}|}
          \hline
          \textbf{Parameter} & \textbf{Value} \\
          \hline
          Material & Carbon Fiber Composite\\
          Rotation Direction	& CCW\\
          Pitch(inch)	& 5\\
          No. of Blades &	2\\
          Hub Diameter (mm)	& 18\\
          Hub Thickness (mm) &	7\\
          Shaft Size (mm) &	9.5/8/6\\
          Adapter Rings (mm) & 6/5/4/3\\
          Weight (g) &	9, (each)\\
          \hline
        \end{tabular}
      \end{table}

      \newpage
      \subsection{Propulsion Powertrain Efficiency}
      The total powertrain involves all the individual components that draw power from the battery, this includes things like the flight controller and flight computer. Here we are interested only in the propulsion powertrain. The propulsion powertrain typically includes: \[ \text{Battery} \quad \quad \rightarrow \quad \quad  \text{ESC} \quad \quad \rightarrow \quad \quad \text{Motors} \]

      The battery and ESC are suppliers, they supply on demand, and since all 4 motors won't derive the same amount of current (and hence power) at the same point of time-- the real-life parameters will vary in time. Here we assume that all motors demand the same power at all times.\\ \\ To quantify the overall efficiency of the UAV’s propulsion system, we analyze losses in each powertrain component. That is mathematically given by, 
      \[ 
      \boldsymbol{\eta_{total} = \eta_{battery} \times \eta_{esc} \times \eta_{motor}}
      \]
      \textbf{Note:} There are other components in the powertrain which we have not discussed here, and which we have discussed later in this report. Here we are only interested in Propulsion powertrain. \\

      In order to calculate each of these components, we would need to calculate the power input and output at each stage. Since we don't currently have access to each component at the moment, we're going to use the parameters provided by the manufacturers for this calculation. \\ \\
      \subsubsection{\large Battery Efficiency Derivation:}
      Given are the following from datasheets:
      \begin{itemize}
        \item Voltage (V): \hfill 11.1V
        \item Max Discharge Current: \hfill 208.0A (40C)
        \item Max Power Output ($ P_{battery} $): \hfill V x I = 11.1 x 208 = \textbf{2308.8 W}
      \end{itemize}
      This output power from the battery shall be used as input to the ESC. Now, to calculate the efficiency of battery, we can define it as, \[ \eta_{battery} = \frac{P_{out}}{P_{stored}} \] But in-flight, it's more feasible to model this using internal resistance. So, 
      \begin{gather*}
        \text{Power lost in battery} = I^2 \, R_{int} \\
        \eta_{battery} = \frac{VI - I^2 R_{int}}{VI} \,\, = \,\, 1 - \frac{IR_{int}}{V}
    \end{gather*}
    Typically, for our battery, the internal resistance is $ \boldsymbol{R_{int} = 0.015 \Omega} $

    \begin{gather*}
      \text{Power loss} = (208)^2 \times 0.015 = 648.96 \, \text{W} \\
      P_{out} = 11.1 \times 208 = 2308.8 \, \text{W} \\
      P_{stored} = 2308.8 + 648.96 = 2957.76 \, \text{W} \\ \\
      \boldsymbol{\eta_{battery}} = \frac{2308.8}{2957.76} \quad \approx \quad 78.07\, \% \quad = \boldsymbol{0.78}
    \end{gather*}
      \subsubsection{\large ESC Efficiency Calculations:}
      The following data from the datasheets:
      \begin{itemize}
        \item Max Continuous Current:	\hfill50A (per channel)
        \item Voltage Range:\hfill	3–6S LiPo (up to 25.2V)
        \item Estimated Losses:\hfill	~5–10\% (heat dissipation)
      \end{itemize}
      Since this is a 4-in-1 ESC, it shares a single power input from the battery and distributes it internally to all 4 ESC channels. The output power is given by,
      \begin{gather*} 
        P_{out} = P_{in} - P_{loss}\\
        \Rightarrow P_{out} = 2308.8 - \frac{5}{100} \times 2308.8 \\
        \therefore P_{out} \,\, \approx \,\, 2193.36 \, W \\ \\
        \Rightarrow \boldsymbol{P_{in} = 2308.8W} \quad \quad \boldsymbol{P_{out} = 2193.36W} \\ 
        \boldsymbol{\eta_{esc}} = \frac{P_{out}}{P_{in}} = \frac{2193.36}{2308.8} \quad \approx \quad \boldsymbol{0.95}
      \end{gather*}
      The total power output shared by all 4-channels of the ESC is \textbf{2193.36W}. A single channel is capable of supplying, \[ \mathrm{P_{in\_motor}} = \mathrm{P_{out\_ESC}} = \frac{2193.36}{4} = \boldsymbol{548.34 \, \text{W}} \]
  

      \subsubsection{\large Motor Efficiency Derivation:}
      The following data is given in the official datasheet:
      \begin{itemize}
        \item KV Rating: \hfill1120 RPM/V
        \item Max Power: \hfill336 W
        \item Max Current: \hfill23.2 A
        \item Voltage Range: \hfill2–4S LiPo (7.4–14.8 V)
        \item Internal Resistance:\hfill 0.070 $\Omega$
        \item Propeller:\hfill 9×5
      \end{itemize}
      The algorithm to derive the motor losses goes as follows: the efficiency is given as,
      \begin{gather*}
        \eta = \frac{\boldsymbol{P_{out}}}{\boldsymbol{P_{in}}}  \\
        \text{Electrical input power:}\quad \boldsymbol{ P_{in} = V \times I} \\
        \text{Mechanical output power:} \quad \boldsymbol{ P_{out} = T \times \omega}
      \end{gather*}
      where \textbf{\textit{V}} is voltage at which thrust is rated, \textbf{\textit{I}} is current drawn at that voltage; \textbf{\textit{T}} is torque generated by the motor (in newton-meters), and $ \boldsymbol{\omega} $ is angular velocity given by, \[ \omega = \frac{2 \pi \times \text{RPM}}{60} \] 
      The RPM without any load will be $ 1120 \times 11.1V = 12432.0 \, rpm $. But when we attach the propellers, some load will be acting agaisnt them, causing the RPM to drop by an amount. Let us assume the new RPM under load is $ \boldsymbol{RPM_{load} = 12000\, rpm} $, then the angular velocity is \[ \omega = \frac{2 \pi \times 12000}{60} \approx 1256.63 \, rad/s\]
  The theoretical torque can be calculated from the formula 
  \begin{gather*} 
    \boldsymbol{\tau = K_t \, . \, I} \\
    \text{where} \quad K_t = \frac{60}{2 \pi \, K_v} \quad and \quad I \rightarrow \text{current in amps} \,\, = \,\, 23.2 \mathrm{A} \\
    \text{now, } \quad \quad K_t = \frac{60}{2 \pi \times 1120} \quad \approx \quad 0.00852 \\ \\
    \therefore \tau = 0.00852 \times 23.2 \, \, = \, \, 0.19780 \, \text{Nm} \\ \\
    \Rightarrow \boldsymbol{ P_{out} \quad = \tau \times \omega = \quad 248.57 \, \text{W} }
  \end{gather*}
  
  Therefore, the motor efficiency is \[ \boldsymbol{\eta_{motor}} = \frac{P_{out}}{P_{in}} = \frac{248.57}{336} = 0.73979 \quad \approx \quad \boldsymbol{0.74} \] which is a pretty reasonable efficiency in real world BLDC motors. Finally, the total propulsion powertrain efficiency is given as, 
  \begin{gather*}
    \eta_{total} = \eta_{battery} \times \eta_{esc} \times \eta_{motor} \\
    \Rightarrow \boldsymbol{\eta_{total}} = 0.78 \times 0.95 \times 0.74 \quad = \quad \boldsymbol{0.55}
  \end{gather*}
  The final propulsion powertrain efficiency sums upto around 55\%, which is a reasonable value considering that some of the parameters we're assumed. Real world values will obviously vary from this.
      
     
  % ======== Aircraft Sizing ========
    \section{Aircraft Sizing}
      \subsection*{Rotor Arm}
      \subsection*{Hub}
      \subsection*{Wheelbase}
      \subsection*{Propeller Clearance}
      \subsection*{Landing Gear}
        
  % ======== Aircraft Performance ====
    \section{Aircraft Performance}
      \subsection{Battery Selection and Endurance}
      \subsubsection{\large Battery: Orange Pro-Range 11.1V 5200mAh (3S)} 
      \begin{wrapfigure}{r}{0.3\textwidth}
        \includegraphics[width=1\linewidth]{battery.png}
        \caption{Orange 11.1V 5200mAh 3S}
        \label{fig:battery3s}
      \end{wrapfigure}
     
      The Orange Pro-Range 11.1V 5200mAh battery is the best for our use case for the following reasons: \\

      The 3S variant provides 11.1V, and has a \textbf{discharge-rate} of 40C. According to the official rated specifications, the maximum continuous discharge current is \textbf{208.0A} (\textit{40C}). It also has a max. burst discharge of \textbf{416.0A} (\textit{80C}). Let us assume that each motor draws 24A current at full-throttle, total current draw would be $ \boldsymbol{24 \times 4 = 96A} $ then \vspace{0.5cm} \[ \text{Theoretical Flight Time (hrs)} = \frac{\text{Capacity (Ah)}}{\text{Current Draw (A)}} = \frac{5.2}{96} \approx 0.0542 hrs = 3.25 mins \] But in real world applications, we dont use 100\% of the battery, we use about 60\%, so that would make the flight time around 5.2mins. 
      \begin{table}[h!]
      \centering
      \caption{Battery Datasheet (Orange 11.1V 5200mAh 3S)}
        \begin{tabular}{|l|c|}
          \hline
          \textbf{Parameter} & \textbf{Value} \\
          \hline
          Voltage & 11.1V (3S) \\
          \hline
          Capacity & 5.2Ah \\
          \hline
          C-Rating (Continuous) & 35C \\
          \hline
          Theoretical Max Discharge Current & $35 \times 5.2 = \mathbf{182A}$ \\
          \hline
          Stated Max Discharge Current (datasheet) & 156A \\
          \hline
          System Current Draw (4 motors @ 24A) & $4 \times 24 = \mathbf{96A}$ \\
          \hline
              Flight Time @ Full Throttle (96A) & $\frac{5.2}{96} = \mathbf{3.25 \text{ minutes}}$ \\
          \hline
          Flight Time @ Moderate Throttle (60A) & $\frac{5.2}{60} = \mathbf{5.2 \text{ minutes}}$ \\
          \hline
          Energy Capacity & $11.1 \times 5.2 = \mathbf{57.72\,Wh}$ \\
          \hline
        \end{tabular}
      \end{table}
    
      \subsection{Total Power Budget Summary}
      This sub-section summarizes the total electrical power budget of the UAS, highlighting how power from the battery is allocated to propulsion and non-propulsion (avionics and payload) subsystems.
      \begin{table}[h!]
        \centering
        \caption{Power Distribution Summary}
        \begin{tabular}{|l|p{5.5cm}|c|c|}
          \hline
          \textbf{Subsystem} & \textbf{Included Components} & \textbf{Power Demand (W)} & \textbf{\% of Total Power} \\
          \hline
          Propulsion & {\small 4 x Motors} & 1344 W & 58.2\% \\
          Avionics & {\small Flight Controller, GPS Module, Sensors (IMU, barometer)} & 6.9486 W & 0.3009 \% \\
          Communication & {\small RC Receiver, Wifi Modules}  & 20.41 W & 0.884 \% \\
          Payload & {\small 2 x Cameras} & 3.879 W & 0.168 \% \\
          Onboard Computer  & {\small NVIDIA Jetson Orin Nano 8GB Module} & 6.997 W & 0.303 \% \\
          \hline
        \end{tabular}
      \end{table}

        \subsubsection{\large Propulsion Demand}
        Based on the datasheet specification and text results, the propulsion subsystems (\textbf{\textit{4xMotors}}) demands \textbf{1344W} during peak operation. This forms the largest share of the total Power requirement ($ \sim $ 58.2 \%).
        \subsubsection{Avionics Demand}
        The avionics system forms the central nervous system of any unmanned aerial vehicle (UAV), including those designed for \textbf{Aerothon-class missions}. Its architecture and power requirements play a decisive role in shaping the overall energy distribution and electrical resilience of the aircraft. The avionics suite typically comprises a flight controller (e.g., Pixhawk), GPS module, telemetry and communication transceivers, RC receiver, and sensor systems such as IMUs and barometers. \\ 

        In advanced UAV configurations—such as ours—it is further augmented by an onboard companion computer, namely the NVIDIA Jetson Orin Nano 8GB, which performs real-time perception and decision-making tasks. \\ 

        Based on our subsystem-level analysis, the avionics block—comprising flight control, communication, and onboard computing—demands a cumulative peak power of approximately \textbf{34.36 W}. This includes:
        \begin{itemize}
          \item \textbf{6.948 W} for the flight controller and embedded sensor suite
          \item \textbf{20.41 W} for communication subsystems including RC receivers and WiFi modules,
          \item \textbf{6.997 W} for the Jetson Orin Nano, which manages AI workloads and perception.
        \end{itemize}
        While this represents only \textbf{1.49 \%} of the total system power draw, the avionics demand is non-negotiable and continuous, requiring high uptime and precision. Power is supplied via a regulated DC bus derived from the main propulsion battery, with dedicated buck converters providing clean and stable 5V and 3.3V rails to sensitive electronics.

        \subsubsection{\large Margins, Safety Factors}
        \textbf{\large Power Budget Margin:} \\
        \begin{enumerate}
          \item \textbf{Purpose:} To ensure that the chosen power source (battery) and propulsion system can consistently provide sufficient power, even under demanding conditions or if components don't perform exactly to spec.
          \item \textbf{Typical Range:} 15\% - 30\% of the calculated total power demand.
          \item \textbf{Application:} After calculating the power required for propulsion (hover, cruise, max thrust) and avionics, an additional percentage is added. 
            \[ \text{Total Required Power = (Propulsion Power + Avionics Power) * (1 + Power Margin)} \]
          \item \textbf{Considerations:} 
            \begin{itemize}
              \item Battery degradation over time.
              \item Variation in motor/propeller efficiency.
              \item Ambient temperature effects on battery performance.
              \item Increased power draw due to wind or aggressive maneuvers.
            \end{itemize}
        \end{enumerate}

        \noindent \textbf{\large Weight/Payload Margin:} \\
        \begin{enumerate}
          \item \textbf{Purpose:} To allow for small increases in component weight during design evolution, manufacturing variations, or for potential future upgrades/additional payloads.
          \item \textbf{Typical Range:} 10\% - 20\% of the estimated total weight (Empty Weight + Max Payload).
          \item \textbf{Application:} When defining the Maximum Take-Off Weight (MTOW), a buffer is included. This also impacts thrust-to-weight ratio calculations. \[ \text{Max All-Up Weight (MAUW) = (Estimated Empty Weight + Max Payload) * (1 + Weight Margin)} \]
          \item \textbf{Considerations:}
            \begin{itemize}
              \item Small design changes or additions.
              \item Manufacturing tolerances.
              \item Unforeseen weight of cabling, fasteners, etc.
            \end{itemize}
        \end{enumerate}
        \noindent \textbf{\large Battery-Capacity/Flight-Time Margin:} \\
        \begin{enumerate}
          \item \textbf{Purpose:} To ensure sufficient energy is available for the planned mission duration, plus a reserve for unexpected events (e.g., strong headwind, holding pattern, emergency landing).
          \item \textbf{Typical Range:} 15\% - 25\% of the total mission energy requirement.
          \item \textbf{Application:} After calculating the energy needed for the mission profile, additional capacity is added. Also, a "return to home" or "emergency landing" battery percentage is typically set (e.g., 20-30\% remaining). 
            \[ (\text{Battery-Capacity})_{\text{Total}} = \frac{ \text{Mission Energy Requirement} }{ \text{Battery Discharge Efficiency} } \times (1 * \text{Capacity Margin}) \] 
          \item \textbf{Considerations:}
            \begin{itemize}
              \item Battery performance degradation over cycles and temperature.
              \item Unexpected mission deviations.
              \item Wind conditions requiring higher power.
              \item Maintaining a safe reserve for landing.
            \end{itemize}
        \end{enumerate}
        \noindent \textbf{\large Structural/Safety Margin:} \\
        \begin{enumerate}
          \item \textbf{Purpose:} To ensure that the drone's airframe and structural components can withstand expected and unexpected loads without failure.
          \item \textbf{Ultimate Factor of Safety (FoS):} 1.5 - 2.0 (or higher for critical components). This is the ratio of ultimate load capacity to the maximum expected operating load.
          \item \textbf{Yield Factor of Safety:} 1.1 - 1.25. This is the ratio of yield strength to the maximum expected operating load, ensuring no permanent deformation.
          \item \textbf{Application:} Applied to material strength calculations for frame arms, motor mounts, landing gear, etc. \[ \text{Required Strength = Maximum Expected Load * Factor of Safety} \]
          \item \textbf{Considerations:}
            \begin{itemize}
              \item Dynamic loads during flight (acceleration, turns).
              \item Impact loads during hard landings or minor crashes.
              \item Vibration fatigue.
              \item Material imperfections and manufacturing variability.
            \end{itemize}
        \end{enumerate}

        

  %========== Material Selection ========
    \section{Material Selection}
      \subsection{Structural Frame, Airframe Components}
    

  %========= Ground Control
    \section{Avionics Subsystems Selection}
      \subsection{Detailed Component Breakdown}
        \subsubsection{\large Flight Controller (Pixhawk 2.4.8)}
        \begin{wrapfigure}{r}{0.3\textwidth}
          \includegraphics[width=1\linewidth]{pixhawk.png}
          \caption{Pixhawk 2.4.8}
          \label{fig:pixhawk}
        \end{wrapfigure}
        Pixhawk is widely egarded as one of the best flight controllers for drone and autonomous aircraft projects — especially in academic and research-grade prototypes — for several compelling reasons:
      \begin{itemize}
        \item \textbf{\textit{Open-Source and Flexible:}} It is built on open hardware and supported by powerful open-source firmware like PX4 or ArduPilot. This enables deep customization, ideal for research and control system testing. And for this reason also, we have \textbf{ample documentation backed by a strong collaborative community, forums, and tutorials}.
        \item \textbf{\textit{Rich I/O capabilities:}} Multiple UART, I2C, CAN, and PWM ports for connecting sensors (GPS, IMU, barometer, etc.) and actuators (ESCs, servos). Ideal for integration with multiple onboard systems including companion computers (e.g., Jetson Nano).
        \item \textbf{\textit{Compatible with autonomous and GPS-guided missions:}}  Supports autonomous navigation, geofencing, waypoints, and RTL (Return to Launch).
        \item \textbf{\textit{Built-in failsafes and safety features:}} Battery failsafes, signal loss handling, and software watchdogs protect the aircraft during unexpected conditions.
        \item \textbf{\textit{Excellent simulation support:}} Compatible with \textbf{HITL (Hardware-in-the-loop)} and \textbf{SITL (Software-in-the-loop)} for control testing and simulation.
      \end{itemize}
      \begin{table}[h!]
        \centering
        \caption{Flight Controller Datasheet (Pixhawk 2.4.8)}
        \begin{tabular}{|l|p{6cm}|}
          \hline
          \textbf{Parameter} & \textbf{Value} \\
          \hline
          Input Voltage (V)	& 7V \\
          Firmware	   &   Mission Planner \\
          Sensor	 &  {\small 3-Axis Gyrometer, Accelerometer, High-performance Barometer, Magnetometer} \\
          Processor	 &    {\small 32bit STM32F427 Cortex M4 core with FPU, The 32-bit STM32F103 failsafe Co-processor} \\
          Micro SD Card Slot (Y/N)	    &   Yes \\
          Dimensions {\small (L x W x H) mm}	   &   82 x 50 x 16 \\
          Weight (g)	 &    40 \\
          \hline
        \end{tabular}
      \end{table}

      \subsubsection{\large Flight Computer (NVIDIA Jetson Orin Nano 8GB)}
      The NVIDIA Jetson Orin Nano 8GB is a powerful, compact AI computing module designed for edge AI applications that demand both high performance and energy efficiency. In the context of our Aerothon UAV project, the Orin Nano plays a pivotal role in enabling advanced onboard computation, particularly for tasks such as \textbf{real-time image processing, autonomous navigation, and object detection.} \\

      \begin{figure}[H]
        \centering 
        \includegraphics[width=0.5\textwidth]{jetson.png}
        \caption{NVIDIA Jetson Orin Nano 8GB Module}
        \label{fig:jetson}
      \end{figure}

      We selected this module not only for its impressive up to \textbf{40 TOPS} of AI performance but also for its \textbf{low power footprint}, which makes it ideal for flight-based applications where every gram and watt matter. The 8GB RAM ensures sufficient memory for running heavy models, such as convolutional neural networks for visual recognition or SLAM algorithms for path planning. \\

      The Jetson Orin Nano interfaces seamlessly with the Pixhawk flight controller via UART or serial USB connections, enabling a tight coupling between autonomous decision-making and low-level control. For example, live video feed from an \textbf{ESP32-CAM} or other camera modules is processed onboard the Jetson, where the output — such as target coordinates or navigation commands — is relayed to the Pixhawk for actuation. \\

      This configuration allows the aircraft to function \textbf{autonomously even without constant ground station communication}, which is critical in GPS-denied or communication-constrained environments. By offloading high-level intelligence to the Jetson module, we achieve a modular and scalable architecture that separates perception and decision-making from flight stabilization, thereby improving system robustness and flexibility. \\

      In essence, the NVIDIA Jetson Orin Nano 8GB empowers our drone with a true edge-AI brain — transforming it from a remotely controlled vehicle into a \textbf{fully autonomous aerial system} capable of intelligent flight and mission execution.

      \subsubsection{Camera}
      \subsubsection{\large Transmitter: (SKYDROID T10 2.4Ghz 10CH)}
      \begin{wrapfigure}{r}{0.3\textwidth}
          \includegraphics[width=1\linewidth]{transmitter.png}
          \caption{Skydriod T10 2.4Ghz}
          \label{fig:skydriod}
        \end{wrapfigure}

      Reasons why this transmitter is good for our mission:
      \begin{itemize}
        \item \textbf{\textit{10-Channel Support:}} Enables simultaneous control of multiple subsystems—throttle, yaw, pitch, roll, camera gimbal, payload, etc.—with high precision.
        \item \textbf{\textit{Long-Range Communication (upto 10-20 kms):}} Essential for long-distance BVLOS (Beyond Visual Line of Sight) flights, ideal for mapping, surveying, or package delivery missions.
        \item \textbf{\textit{Low Latency, High Reliability:}} Ensures stable control inputs with low delay (<200 ms), which is critical for real-time maneuvering and emergency responses.
        \item \textbf{\textit{Telemetry Integration:}} Supports two-way telemetry for critical in-flight data monitoring such as battery voltage, GPS, altitude, and more.
        \item \textbf{\textit{Open Protocol Compatibility:}} Works seamlessly with systems like ArduPilot and PX4, making it a perfect match for autopilots like Pixhawk.
      \end{itemize}

      \begin{table}[h!]
        \centering
        \caption{Transmitter Datasheet (Skydroid T10 2.4Ghz)}
        \begin{tabular}{|l|p{6cm}|}
          \hline
          \textbf{Parameter} & \textbf{Value} \\
          \hline
          Frequency (GHz):	&    2.4 \\
          Operating Voltage (V):	&    4.2V \\
          Working Current (mA)	&   100mA \\
          Dimensions (L x W x H) mm	&      150*130*20 \\
          Weight (g):	 & 525g      \\
          \hline
        \end{tabular}
      \end{table}

       The T10-R10 combo is designed for mission-critical UAV applications. They reduce system complexity by combining control, video, and telemetry into a single ground control unit. Ideal for mapping, surveillance, agricultural spraying, or delivery drones—as in Aerothon. Seamlessly integrates with Pixhawk flight controllers, NVIDIA Jetson modules, and other onboard avionics.


      \subsubsection{\large Receiver: Skydriod R10}
      \begin{itemize}
        \item \textbf{\textit{10 PWM Channels or SBUS/PPM Support:}} Can directly interface with flight controllers (like Pixhawk) and handle multiple channels via a single signal wire (SBUS).


        \item \textbf{\textit{High-Speed, Low-Latency Link:}} Matches the T10’s low-latency protocol for precise motor response and stability.
        \item \textbf{\textit{Secure Signal Handling:}} Supports frequency-hopping and error correction to minimize interference and signal loss, even in high-noise RF environments.
        \item \textbf{\textit{Long-Range Capability:}} Complements the T10’s range, enabling safe long-range flight and robust signal quality even at maximum distance.
        \item \textbf{\textit{Powerful Telemetry Interface:}} Allows downlink of data to the transmitter (e.g., battery status, GPS info), helping pilots make informed decisions mid-flight.
      \end{itemize}




    \subsection{Avionics Powertrain Efficiency}

    \section{Autonomous Navigation System}
      \subsection{Hardware Setup}
      \subsection{Software Architechture}

    \section{C.G. Calculation \& Stability Analysis}
      \subsection{Lift, Drag and Stability Considerations}
      \subsection{Center of Gravity Position \& Trim}

  % =========== Computationl Analysis ============
  \chapter{Computational Analysis}
    \section{CFD / FEM / MATLAB Simulations}
    \section{CAD Model and Performance Validation}
  % =========== Safety, Compliance & Risk Assessment====== 
  \chapter{Safety \& SORA Assessment}
    \section{Risk Analysis and Mitigation Strategies}
      
  % ========= Conclusion & Future Work =====
  \chapter{Methodology for Autonomous Operations}
    \section{Flight Control Algorithm}
    \section{Object Detection \& Counting}
    \section{Autonomous Payload Drop Mechanism (Gripper)}

  \chapter{Innovations and Future Scope}
  \chapter{Bill of Materials}
    \begin{table}[h!]
      \centering
      \caption{Bill of Materials} \vspace{0.2cm}
        \begin{tabular}{|l|c|c|}
          \hline
          \textbf{Component Name} & \textbf{Quantity} & \textbf{Unit Price {INR}} \\
          \hline
          NVIDIA Jetson Orin Nano 8GB & 1 & 36,499 \\
          Pixhawk 2.4.8 Flight Controller & 1 & 11,179 \\
          Cameras & 2 & 1,000 \\
          SpeedyBee BL32 50A 4-in-1 ESC &1 &7139\\
          DYS D2836-7 1120KV Brushless Motor &4 &1,228\\
          Orange Pro-Range 5200mah 11.1V & 1 &3,653\\
          GPS – Neo M8N  &1 &1,500\\
          SkyDroid Transmitter  &1 & 12,075\\
          Servo & 1& 500\\
          Frame  &1& 3,000\\
          9-inch Propellers & 4 &200\\
          \hline
          \textbf{Total}  & --  & \textbf{83,257 /-}   \\      
          \hline
        \end{tabular}
    \end{table}

\end{document}

