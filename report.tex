\documentclass[12pt]{report}
\usepackage[a4paper, margin=0.5in]{geometry}
\usepackage{graphicx, wrapfig, hyperref, amsmath, amssymb, geometry, array, float}
\renewcommand{\familydefault}{\sfdefault}
\usepackage[scaled]{helvet}
\usepackage[table]{xcolor}
\geometry{margin=1in}


\usepackage{titlesec}

\titleformat{\chapter}[display]
  {\normalfont\huge\bfseries}{}{0pt}{\Huge\vspace{-2cm}} % you can tweak this

\titlespacing*{\chapter}{0pt}{5pt}{20pt} % left, before, after


\begin{document}

% ====================== TITLE ========================
  \begin{center}
    % \vspace*{-1in}
    \includegraphics[width=0.4\textwidth]{sae_aerospace.png}\\
    \vspace{1cm}
    \textbf{\LARGE AEROTHON 2025} \\
    \vspace{0.5in}
    \textbf{\Large UNCREWED AIRCRAFT SYSTEM  (UAS) \\
    DESIGN, BUILD \\ AND FLY CONTEST}\\
    \vspace{0.5cm}
    \includegraphics[width=0.4\textwidth]{aerothon2025.png}\\
    \vspace{0.5cm}
    \Large \textbf{PHASE 1:} DESIGN REPORT \\
    \vspace{0.5cm}
    \Large \textbf{TEAM NAME:} UDSAV \\
    \Large \textbf{TEAM NUMBER: } AT2025043
    
    \vspace{0.5cm}
    \includegraphics[width=0.3\textwidth]{gsvlogo.png}\\
    \vspace{0.5cm}
    \textbf{\Large GATI SHAKTI VISHWAVIDYALAYA} \\
    \large{Ministry of Railways, Govt. of India \\Vadodara, India - 390004}
  \end{center}
  % ======================================================
  \newpage
  add certificate faculty mentor 
  \newpage

  \newgeometry{left=1in, right=0.5in, top=1in, bottom=0.5in}
  % ============ [TABLE OF CONTENTS] ================
  \tableofcontents
  \newpage

  \newgeometry{left=1in, right=0.5in, top=0.5in, bottom=0.5in}

  % ================== Introduction =================
  \chapter{Introduction}
  \section{Overview}

   In the face of natural and man-made disasters, rapid response and situational awareness are critical. Drones have emerged as powerful tools in disaster management, offering real-time aerial insights, access to hard-to-reach areas, and faster deployment compared to traditional methods. Whether locating survivors, assessing damage, or delivering essential supplies, drone technology enhances the efficiency and safety of relief operations. As disasters grow more complex and unpredictable, integrating drones into emergency response systems is no longer a luxury—it's a necessity.

    \section{Problem Statement and Mission Requirements}
    This year's AEROTHON is themed on \textbf{\textit{Surveillance and Disaster Management.}} The problem statement is to build an \textbf{\textit{Uncrewed Aircraft System (UAS)}} to be able to perform the mission requirements as per the rulebook. The mission requirements at a glance are as follows: \\ \\
    \textbf{Mission - 1:} \textit{Advanced Obstacle Navigation $\&$ Fragile Payload Delivery with Precision Placement – Manual Operation} \\ \\
    \textbf{Mission - 2:} \textit{Autonomous Object Classification, Disaster Situation Identification $\&$ Payload Drop – Autonomous Operation} 



    \section{Scope of Report}
    The scope of this report is to provide a comprehensive understanding of the design rationale we have used while building this project. We have tried to provide the relevant calculations, figures, and analysis models to justify the materials/design/framework we've chosen to work with for our structural and system architectures. \\ 

    \noindent Apart from that, this report is intended to also serve as an accessible guide catering to neophytes in UAV/UAS systems. We have tried our best to aim at providing clear context and insight that sort of demystifies drone development.

  % ======== Requirements and Design Objectives =========
  \section{System Requirements \& Design Objectives}
    \subsection{Mission Profile}
    \begin{enumerate}
      \item \textbf{Mission 1:} \textit{Advanced Obstacle Navigation $\&$ Fragile Payload Delivery with Precision Placement} \\
        This is a \textbf{\textit{Manual Operation.}} In this mission, the drone must transport a fragile payload through a challenging course filled with static obstacles such as walls, barriers, and narrow passages. The primary objective is to navigate these obstacles with high precision while ensuring the payload remains undamaged. \\
        
        Upon reaching the target zone, the drone must land carefully and place the fragile payload on the ground without causing any damage. After the successful placement, the drone must then return to the takeoff point or designated home base, ensuring safe and efficient navigation back through the course. The mission is complete once the payload is placed securely, and the drone successfully returns to the home base.
      \item \textbf{Mission 2:} \textit{Autonomous Object Classification, Disaster Situation Identification $\&$ Payload Drop }\\
        This is an \textbf{\textit{Autonomous Operation.}} In this mission, the drone will autonomously scan, classify, and assess objects within a predefined area using onboard sensors and algorithms. The objects will vary in shape, size, color, and structure, and may be partially obscured, presenting challenges for detection and classification. Once the objects are classified, the drone will identify potential disaster scenarios, such as flooding, fire, or damaged infrastructure, within the same area.
    \end{enumerate}
    \subsection{Key Performance Indicators \& Constraints}
    According to the above defined mission profiles, we have a few KPIs (\textit{Key Performance Index}) to keep in mind.
    \begin{enumerate}
      \item Flight Endurance and Range 
      \item Payload Handling 
      \item Autonomous Capabilites
      \item System Reliability
      \item Design and Innovation
    \end{enumerate}
        The design and development of the UAV is subjected to several constraints as per the guidelines mentioned in the rulebook AEROTHON 2025. These include dimensional constraints, payload restrictions and strict autonomy requirements. The drone must perform all missions bound by these constraints and we have taken great time and care to articulate them down to ensure nothing is amiss.
    \begin{enumerate}
      \item \textbf{Dimensional Constraints}
        \begin{itemize}
          \item Maximum Wingspan: \textbf{1.5 metres} - the UAV must fit inside a \textbf{\textit{1.5m x 1.5m x 1.5m bounding box}} in assembled condition.
          \item Maximum Takeoff Weight: \textbf{3.5kg} including battery and payload.
        \end{itemize}
      \item \textbf{Payload Constraints}
        \begin{itemize}
          \item Payload: One fragile payload cube of \textbf{\textit{10cm x 10cm x 10cm}} weighing \textbf{\textit{150 ~ 200g}}.
          \item Payload must be released within a \textbf{\textit{3m x 3m}} target zone.
        \end{itemize}
      \item \textbf{Flight Environment Constraints}
        \begin{itemize}
          \item Missions are conducted in \textbf{\textit{open outdoor airspace}}.
          \item Expect wind speeds upto \textbf{\textit{5m/s}}
        \end{itemize}
      \item \textbf{Autonomy and Mission Constraints}
        \begin{itemize}
          \item \textbf{Mission 1:} Manual flight only (no GPS or autopilot usage).
          \item \textbf{Mission 2:} Fully autonomous flight (no pilot intervention or RC use).
          \item All autonomous missions must avoid obstacles and make decisions based on \textbf{\textit{onboard computation.}}
        \end{itemize}
      \item \textbf{Power and Communication Constraints}
        \begin{itemize}
          \item Must operate on battery only
          \item No cellular or internet-based comms allowed
          \item Only 2.4 GHz or 5.8 GHz RF modules permitted
        \end{itemize}
      \item \textbf{Safety and Compliance}
        \begin{itemize}
          \item Must have a failsafe mode (e.g., return-to-home or emergency land)
          \item Must pass technical inspection before flying
          \item Compliance with DGCA drone guidelines (if relevant in test zones)
        \end{itemize}
      \item \textbf{Operational Constraints}
        \begin{itemize}
          \item The team must complete the flight within a \textbf{\textit{15-minute slot.}}
          \item Payload must be dropped in an area of \textbf{\textit{3m x 3m.}}
        \end{itemize}
    \end{enumerate}
  % ========= Conceptual Design Approach =======
  \chapter{Conceptual Design Approach}
    \section{Design Methodology}
    \section{Product Benchmark \& Trade-off Analysis} % why we chose this design over other 
                                                 
  % ======== Aerodynamic and Flight Performance ======
  \chapter{Detailed Design Breakdown}                                 
    \section{Preliminary Weight Estimation}

  % ============== Propulsion ====================
    \section{Thrust Requirement \& Propulsion System Selection}
      \subsection{Motor, ESC, Propellor \& Battery Selection}
      \vspace{0.5cm}
      
      \textbf{Motor: DYS D2836-7 1120KV Brushless Motor} \\
      \begin{wrapfigure}{r}{0.2\textwidth}
        \includegraphics[width=0.2\textwidth]{bldc.png}
        \caption{DYS D2836-7 1120kV BLDC}
        \label{fig:bldc1120}
      \end{wrapfigure}
      The DYS D2836-7 1120KV Brushless Motor is our go-to motor for this project because of the following leverages it offers:
        
      \begin{itemize}
        \item \textbf{\textit{KV Rating: 1120KV}} - KV generally means RPM per volt. In layman terms, in one volt, how many rotations does it make per minute = KV. In this case, 1120KV is mid-range, which means good thrust at moderate RPMs, and it works decently with larger propellors (\textit{9" - 11"}) which improves lift and efficiency, \textit{especially} at low speeds. This is perfect for surveillance drones that require loitering and stability. A lower KV would force us to use bulky propellors, and a higher KV would drain the battery faster. 1120KV is a sweet spot between the two.
        \item \textbf{\textit{Power \& Efficiency:}} - With a 3S or 4S LiPo, this motor produces ~800g to 1100g of thrust, depending on the propeller used. It can pull 20–25A max, so it's efficient for mid-weight UAVs (\textit{in our case, it is around \textbf{1.5 ~ 2kg AUW} (All Up Weight.)}), so it's ideal for our choice.
      \end{itemize}
      \renewcommand{\arraystretch}{1.5}
      \rowcolors{1}{white}{blue!10}
      
      \begin{table}[H]
      \centering
      \caption{Motor Characteristics (DYS D2836-7 1120kV BLDC)}
        \begin{tabular}{|>{\raggedright\arraybackslash}p{6cm}|>{\raggedright\arraybackslash}p{6cm}|}
          \hline
          \textbf{Parameter} & \textbf{Value} \\
          \hline
          Motor KV (RPM/V) & 1120 \\
          \hline
          Motor Type & Brushless Motor \\
          \hline
          Compatible LiPO Batteries & 2S to 4S \\
          \hline
          Weight (g) & 70 \\
          \hline
          Shaft Diameter (mm) & $\Phi$4.0×49mm \\
          \hline
          Max. Power (W) & 336 \\
          \hline
          Maximum Thrust (gm) & 1130 \\
          \hline
          Compatible Prop (inch) & 11×7 / 7×3 \\
          \hline
          Required ESC (A) & 40 \\
          \hline
          Shipping Weight & 0.089 kg \\
          \hline
          Shipping Dimensions & 10 × 6 × 5 cm \\
          \hline
        \end{tabular}
      \end{table}
      \vspace{0.5cm}
      \textbf{ESC: ReadytoSky 40A 2-4S ESC} \\
      \begin{wrapfigure}{r}{0.3\textwidth}
        \includegraphics[width=1\linewidth]{esc.png}
        \caption{ReadytoSky 40A 2-4S ESC}
        \label{fig:esc40a}
      \end{wrapfigure}

      The ReadytoSky 40A 2-4S ESC is a good choice for surveillance drones, and our use case for the following reasons:
      \begin{itemize}
        \item \textbf{\textit{Sufficient Current Handling}} - Our DYS D2836 motor draws about 20 -- 25A at max. The ESC we're using can supply upto 40A, which is comfortably safe for our use case.
        \item \textbf{\textit{2-4s LiPo Compatiility}} - This works best with common drone battery setups like 3S -- 4S. Another reason for this choice is that it matches our motor's voltage tolerance (D2836 runs fine upto 4S).
        \item \textbf{\textit{BLHeli Firmware}} - Comes pre-flashed  with BLHeli, and this offers a smooth throttle response; brake support (for fixed wing or quad hybrid designs).
        \item \textbf{\textit{Thermal Performance and Build}} - The ESC is rating to suport 40A, which means it can take bursty loads without frying good for maneuvers, sudden wind corrections, or takeoffs.
      \end{itemize}

      \vspace{0.5cm}
      \noindent \textbf{Propellor: } 

      \vspace{0.5cm}
      

      \subsection{Powertrain Efficiency Calculations}
  % ======== Aircraft Sizing ========
    \section{Aircraft Sizing}
      \subsection*{Rotor Arm}
      \subsection*{Hub}
      \subsection*{Wheelbase}
      \subsection*{Propellor Clearance}
      \subsection*{Landing Gear}
        
  % ======== Aircraft Performance ====
    \section{Aircraft Performance}
      \subsection{Power Estimation}
      \subsection{Battery Selection and Endurance}
      \noindent \textbf{Battery: Orange Pro-Range 11.1V 5200mAh (3S)} \\
      \begin{wrapfigure}{r}{0.3\textwidth}
        \includegraphics[width=1\linewidth]{battery.png}
        \caption{Orange 11.1V 5200mAh 3S}
        \label{fig:battery3s}
      \end{wrapfigure}
     
      The Orange Pro-Range 11.1V 5200mAh battery is the best for our use case for the following reasons: \\

      The 3S variant provides 11.1V, and has a \textbf{discharge-rate} of 40C. According to the official rated specifications, the maximum continuous discharge current is \textbf{208.0A} (\textit{40C}). It also has a max. burst discharge of \textbf{416.0A} (\textit{80C}). Let us assume that each motor draws 24A current at full-throttle, total current draw would be $ \boldsymbol{24 \times 4 = 96A} $ then \vspace{0.5cm} \[ \text{Theoretical Flight Time (hrs)} = \frac{\text{Capacity (Ah)}}{\text{Current Draw (A)}} = \frac{5.2}{96} \approx 0.0542 hrs = 3.25 mins \] But in real world applications, we dont use 100\% of the battery, we use about 60\%, so that would make the flight time around 5.2mins. 
      \begin{table}[h!]
      \centering
      \caption{Battery and Flight Characteristics Summary (Orange 11.1V 5200mAh 3S)}
        \begin{tabular}{|l|c|}
          \hline
          \textbf{Parameter} & \textbf{Value} \\
          \hline
          Voltage & 11.1V (3S) \\
          \hline
          Capacity & 5.2Ah \\
          \hline
          C-Rating (Continuous) & 35C \\
          \hline
          Theoretical Max Discharge Current & $35 \times 5.2 = \mathbf{182A}$ \\
          \hline
          Stated Max Discharge Current (datasheet) & 156A \\
          \hline
          System Current Draw (4 motors @ 24A) & $4 \times 24 = \mathbf{96A}$ \\
          \hline
          Flight Time @ Full Throttle (96A) & $\frac{5.2}{96} = \mathbf{3.25 \text{ minutes}}$ \\
          \hline
          Flight Time @ Moderate Throttle (60A) & $\frac{5.2}{60} = \mathbf{5.2 \text{ minutes}}$ \\
          \hline
          Energy Capacity & $11.1 \times 5.2 = \mathbf{57.72\,Wh}$ \\
          \hline
        \end{tabular}
      \end{table}


  %========== Material Selection ========
    \section{Material Selection}
      \subsection{Structural Frame, Airframe Components}
    

  %========= Ground Control
    \section{Subystems Selection}
      \subsection{Flight Controller, Sensors, Navigation}
      \subsection{Communication Systems}

    \section{Autonomous Navigation System}
      \subsection{Hardware Setup}
      \subsection{Software Architechture}

    \section{C.G. Calculation \& Stability Analysis}
      \subsection{Lift, Drag and Stability Considerations}
      \subsection{Center of Gravity Position \& Trim}

  % =========== Computationl Analysis ============
  \chapter{Computationl Analysis}
    \section{CFD / FEM / MATLAB Simulations}
    \section{CAD Model and Performance Validation}
  % =========== Safety, Compliance & Risk Assessment====== 
  \chapter{Safety \& SORA Assessment}
    \section{Risk Analysis and Mitigation Strategies}
      
  % ========= Conclusion & Future Work =====
  \chapter{Methodology for Autonomous Operations}
    \section{Flight Control Algorithm}
    \section{Object Detection \& Counting}
    \section{Autonomous Payload Drop Mechanism (Gripper)}

  \chapter{Innovations and Future Scope}
  \chapter{Bill of Materials}
\end{document}

